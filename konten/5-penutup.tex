\chapter{PENUTUP}

\section{Kesimpulan}

Hasil yang diharapkan dari penelitian ini adalah model reidentifikasi yang dibuat pada penelitian ini memiliki nilai akurasi yang lebih tinggi atau setidaknya menyamai penelitian sebelumnya yang menggunakan dataset CCTV. Penulis juga berharap penelitian mengenai reidentifikasi mobil dengan memanfaatkan UAV akan lebih banyak dipelajari oleh para peneliti.

\section{Saran}

Sejauh ini penulis telah menentukan dataset yang akan digunakan dalam penelitian ini. Selain itu, penulis juga sudah mempelajari mengenai \emph{Convolutional Neural Network} (CNN) dan juga membaca penelitian-penelitian terkait sistem reidentifikasi kendaraan berbasis \emph{Convolutional Neural Network} (CNN). Penulis juga sudah mencoba melakukan \emph{training} dataset menggunakan model arsitektur ResNet-50 yang menghasilkan suatu model \emph{deep learning} yang dapat melakukan klasifikasi bunga. Saat ini, penulis sedang melakukan tahap \emph{preprocessing} dataset. Tahap \emph{preprocessing} yang sedang dilakukan yaitu \emph{splitting} dataset. Dalam proses \emph{splitting}, citra-citra pada dataset akan dibagi ke dalam \emph{folder-folder} yang sesuai sehingga akan memudahkan saat proses \emph{training}.
