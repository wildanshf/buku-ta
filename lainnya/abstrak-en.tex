\addcontentsline{toc}{chapter}{ABSTRACT}

\begin{center}
  \large
  \textbf{ABSTRACT}
\end{center}

\vspace{4mm}

\begin{center}
  \textbf{
    VEHICLE REIDENTIFICATION USING CONVOLUTIONAL NEURAL NETWORK (CNN)
    ON UNMANNED AERIAL VEHICLE (UAV) IMAGES
  }
\end{center}

% % Menyembunyikan nomor halaman
% \thispagestyle{empty}

\begin{flushleft}
  \setlength{\tabcolsep}{0pt}
  \bfseries
  \begin{tabular}{lc@{\hspace{6pt}}l}
    Student Name / NRP & : & Wildan Shafa Diandra/ 07211940000049    \\
    Department         & : & Computer Engineering ELECTICS - ITS     \\
    Advisor            & : & 1. Reza Fuad Rachmadi, S.T., M.T., Ph.D \\
                       &   & 2. Dr. I Ketut Eddy Purnama, S.T., M.T. \\
  \end{tabular}
  \vspace{4ex}
\end{flushleft}
\textbf{Abstract}

% Isi Abstrak
In recent years the number of motor vehicle in Indonesia has continued to increase. A large number of surveillance cameras are used on street corners and intersections to assist in the traffic monitoring process. The monitoring system is mostly based on real-time video and still requires humans to monitor it, so it is still ineffective. In addition, the use of surveillance cameras is also less effective when used to monitor multiple locations. One surveillance camera can only be used to monitor 1 location point. Currently, studies on CNN-based car re-identification systems have begun to be developed more. A car re-identification system can help the traffic monitoring process, especially in monitoring, tracking, and finding cars. At the same time, the utilization of Unmanned Aerial Vehicles (UAV) has begun to be widely used in various sectors of life, one of which is in the field of computer vision. In this study, the author wanted to create a CNN model that could carry out a car re-identification process. The dataset that will be used in this study was taken using UAV. The author hopes that with this study, studies on car re-identification system that utilizes UAVs will be studied more by many researchers. The author also hopes that the car re-identification system can be implemented in a traffic monitoring system to make it easier for the staff in charge.

\vspace{2ex}
\noindent
\textbf{Keywords: \emph{Convolutional Neural Network, Car Re-identification, Unmanned Aerial Vehicle}}