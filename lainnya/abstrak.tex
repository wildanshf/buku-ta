\addcontentsline{toc}{chapter}{ABSTRAK}

\begin{center}
  \large
  \textbf{ABSTRAK}
\end{center}

\vspace{4mm}

\begin{center}
  \textbf{
    REIDENTIFIKASI KENDARAAN MENGGUNAKAN \emph{CONVOLUTIONAL NEURAL NETWORK} (CNN)
    PADA CITRA \emph{UNMANNED AERIAL VEHICLE} (UAV)
  }
\end{center}

% % Menyembunyikan nomor halaman
% \thispagestyle{empty}

\begin{flushleft}
  \setlength{\tabcolsep}{0pt}
  \bfseries
  \begin{tabular}{ll@{\hspace{6pt}}l}
    Nama Mahasiswa / NRP & : & Wildan Shafa Diandra / 07211940000049   \\
    Departemen           & : & Teknik Komputer FTEIC - ITS             \\
    Dosen Pembimbing     & : & 1. Reza Fuad Rachmadi, S.T., M.T., Ph.D \\
                         &   & 2. Dr. I Ketut Eddy Purnama, S.T., M.T. \\
  \end{tabular}
  \vspace{4ex}
\end{flushleft}
\textbf{Abstrak}

% Isi Abstrak
Beberapa tahun terakhir jumlah kendaraan bermotor di Indonesia terus meningkat. Sejumlah besar kamera pengintai digunakan di sudut-sudut jalan maupun persimpangan untuk membantu proses pemantauan lalu lintas. Sistem pemantauan tersebut sebagian besar berbasis video real-time dan masih membutuhkan manusia untuk mengawasinya sehingga masih kurang efektif. Selain itu, penggunaan kamera pengintai juga kurang efektif apabila digunakan untuk mengawasi banyak lokasi. Satu buah kamera pengintai hanya bisa digunakan untuk mengawasi 1 titik lokasi. Saat ini penelitian mengenai system reidentifikasi mobil berbasis CNN mulai banyak dikembangkan. Sistem reidentifikasi mobil dapat membantu proses pemantauan lalu lintas khususnya dalam memantau, melacak, maupun mencari mobil. Pada saat yang sama, pemanfaatan Unmanned Aerial Vehicles (UAV) mulai banyak digunakan di berbagai sektor kehidupan, salah satunya di bidang visi komputer. Pada penelitian ini, penulis ingin membuat model CNN yang dapat melakukan reidentifikasi mobil. Dataset yang akan digunakan pada penelitian ini diambil menggunakan UAV. Penulis berharap dengan adanya penelitian ini, penelitian tentang teknologi identifikasi ulang mobil yang memanfaatkan UAV akan lebih banyak dipelajari oleh para peneliti. Penulis juga berharap sistem reidentifikasi mobil dapat diimplementasikan pada sistem pemantauan lalu lintas sehingga memudahkan staf yang bertugas.

\vspace{2ex}
\noindent
\textbf{Kata Kunci: \emph{Convolutional Neural Network, Reidentifikasi Mobil, Unmanned Aerial Vehicle}}